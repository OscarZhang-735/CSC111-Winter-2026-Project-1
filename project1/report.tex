\documentclass[11pt]{article}
\usepackage{amsmath}
\usepackage{amsfonts}
\usepackage{amsthm}
\usepackage[utf8]{inputenc}
\usepackage[margin=0.75in]{geometry}

\title{CSC111 Winter 2026 Project 1}
\author{Chenxi Wu, Jiaming Zhang}
\date{\today}

\begin{document}
\maketitle

\section*{Running the game}
Simply running \texttt{adventure.py}.

\section*{Game Map}
Example game map below:

\begin{verbatim}
-1  3
 4  2
 5  1
\end{verbatim}

Starting location is: 1

\section*{Game solution}

One possible list of commands that leads to winning the game is:

\begin{verbatim}
check phone
go north
take laptop charger
go west
take usb drive
go south
check locker
go north
go east
go north
check password note
go south
go south
check photo album
go west
check locker
take lucky mug
go east
drop usb drive
drop laptop charger
\end{verbatim}


\section*{Lose condition(s)}

The player loses the game if the number of movement commands reaches the maximum allowed limit (20) without returning all required items to the dorm room.

One example list of commands that leads to losing the game is repeatedly moving back and forth until the move limit is reached:

\begin{verbatim}
go north
go south
go north
go south
...
\end{verbatim}

The parts of the code involved in handling this lose condition are:
\begin{itemize}
    \item \texttt{adventure.py}, inside the main game loop
    \item The \texttt{move} attribute in the \texttt{AdventureGame} class
\end{itemize}


% Copy-paste the above if you have multiple lose conditions and describe each possible way to lose the game

\section*{Inventory}

\begin{enumerate}
\item All location IDs that involve items in the game:

Locations 1, 2, 3, 4, and 5 all contain at least one item.

\item Key item data:
\begin{enumerate}
    \item For Item 1:
    \begin{itemize}
        \item Item name: laptop charger
        \item Item start location ID: 2
        \item Item target location ID: 1
    \end{itemize}

    \item For Item 2:
    \begin{itemize}
        \item Item name: usb drive
        \item Item start location ID: 4
        \item Item target location ID: 1
    \end{itemize}

    \item For Item 3:
    \begin{itemize}
        \item Item name: lucky mug
        \item Item start location ID: 5
        \item Item target location ID: 1
    \end{itemize}
\end{enumerate}

\item Exact command(s) used to pick up and drop an item:

\begin{verbatim}
take laptop charger
inventory
drop laptop charger
\end{verbatim}

\item Parts of the code involved in handling the \texttt{inventory} command:
\begin{itemize}
    \item \texttt{adventure.py}
    \item Class: \texttt{AdventureGame}
    \item Methods: \texttt{display\_inventory}, \texttt{take\_item}, \texttt{drop\_item}
\end{itemize}
\end{enumerate}


\section*{Score}

\begin{enumerate}
\item Players can earn score by:
\begin{itemize}
    \item Checking important items such as the phone, password note, locker, and photo album (+10)
    \item Successfully dropping target items in the dorm room (+30)
    \item Moving between locations (+5)
\end{itemize}

The first opportunity to earn score occurs at location 1 by using the command:

\begin{verbatim}
check phone
\end{verbatim}

\item The command list used in the score demonstration is:

\begin{verbatim}
go north
score
go south
\end{verbatim}

\item Parts of the code involved in handling scoring:
\begin{itemize}
    \item \texttt{adventure.py}
    \item Class: \texttt{AdventureGame}
    \item Methods: \texttt{check\_item}, \texttt{drop\_item}
    \item The main game loop where the \texttt{score} command is processed
\end{itemize}
\end{enumerate}

\section*{Enhancements}

\begin{enumerate}
\item Story-based puzzle using flags
\begin{itemize}
    \item Brief description:
    The game includes a story-driven puzzle where the player must gather clues from different items (phone, password note, and photo album) to unlock the locker and retrieve the lucky mug.

    \item Complexity level: medium

    \item Reasons for complexity level:
    This enhancement required adding multiple flags to track player progress, conditional logic when checking items, and coordination between different locations and items to unlock the puzzle.

    \item Parts of the code involved:
    \begin{itemize}
        \item \texttt{adventure.py}
        \item Class: \texttt{AdventureGame}
        \item Method: \texttt{check\_item}
    \end{itemize}

    \item Demonstration command list:

\begin{verbatim}
check phone
go north
go north
check password note
go south
go south
check photo album
check locker
\end{verbatim}

\end{itemize}
\end{enumerate}


    % Uncomment below section if you have more enhancements; copy-paste as needed
    %\item Describe your enhancement here
    %\begin{itemize}
    %    \item Basic description of what the enhancement is:
    %    \item Complexity level (low/medium/high):
    %    \item Reasons you believe this is the complexity level (e.g., mention implementation details)
    %    \item Name the parts of the code which are involved in this enhancement
    %    \item Copy the list you assigned to \texttt{enhancements\_demo} in the \texttt{simulation.py} file into this section of the report:
    %\end{itemize}
\end{enumerate}


\end{document}
